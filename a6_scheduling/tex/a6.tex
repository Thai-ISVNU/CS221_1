\documentclass[12pt]{article}
\usepackage{fullpage,enumitem,amsmath,amssymb,graphicx, listings, array, bbm}

\begin{document}

\begin{center}
{\Large CS221 Fall 2018 - 2019 Homework 6}

\begin{tabular}{rl}     
Name: & Dat Nguyen \\
Date: & 3/20/2019
\end{tabular}
\end{center}
 
 By turning in this assignment, I agree by the Stanford honor code and declare that all of this is my own work.

\section*{Problem 0: Warmup}
\begin{enumerate}[label=(\alph*)]
	\item
	We let the variable $X_1, X_2,\dots, X_m \in {0, 1}$ to be whether or not we press button j = 1,\dots,m \\
	Let the constraint for bulb i = 1,\dots,n be
	\begin{align*} 
		f_i(T) = \mathbbm{1}[(-1)^{\sum_{j \in 1,\dots,m} X_j\mathbbm{1}[\text{bulb}_i \in T_j]} = 0] 
	\end{align*}
	Scope of each constrain is all variables $X_1, X_2,\dots, X_m$

	\item
	\begin{enumerate}[label=\roman*]
	\item
	There are 2 consistent assignments for this CSP, namely $X_1 = 1, X_2 = 0, X_3 = 1$ and $X_1 = 0, X_2 = 1, X_3 = 0$.
	\item
	Let D be the original domain, the call stack of backtrack() is as follow \\
	backtrack($\emptyset$, 1, D) \\
	backtrack($\{X_1 = 0\}$, 1, D) \\
	backtrack($\{X_1 = 0, X_3 = 0\}$, 1, D) \\
	backtrack($\{X_1 = 0, X_3 = 0, X_2 = 0\}$, 0, D) \\
	backtrack($\{X_1 = 0, X_3 = 0, X_2 = 1\}$, 1, D)
	\item
	backtrack($\emptyset$, 1, D) \\
	backtrack($\{X_1 = 0\}$, 1, $\text{Domain}_1 = \{1\}, \text{Domain}_2 = \{1\}, \text{Domain}_3 = \{0\}$) \\
	backtrack($\{X_1 = 0, X_3 = 0\}$, 1, $\text{Domain}_1 = \{1\}, \text{Domain}_2 = \{1\}, \text{Domain}_3 = \emptyset$) \\
	backtrack($\{X_1 = 0, X_3 = 1, X_2 = 0\}$, 1, $\text{Domain}_1 = \{1\}, \text{Domain}_2 = \emptyset, \text{Domain}_3 = \emptyset$) \\

	\end{enumerate}
\end{enumerate}

\section*{Problem 2: Handling n-ary factors}
\begin{enumerate}[label=(\alph*)]
	\item 
	For each variable $X_i$ we introduce an auxiliary variable $A_i$ which keeps track of the sum of all variables before $X_i$. The domain of $A_i$ is a tuple which first element is possible value of sum ($\{0, 1, 2, 3, 4, 5, 6\}$) and second element is possible value of $X_i$ ($\{0, 1, 2\}$) denoting the value to be included to the sum at this step. The constraint for $A_i$ and $X_i$ is that the second element of the tuple by $A_i$ must be consistent with value of $X_i$. In addition, for two consecutive steps we ensure that the sum of 2 elements in the tuple of $A_i$ to be equal to the first element of the tuple of the next step $A_{i+1}$. The final step we will have a variable 'result' which is consistent with the sum of all $X_i$ and we add one more constraint that its value is $\leq K$.
\end{enumerate}

\section*{Problem 3: Course Scheduling}
\begin{enumerate}[label=(\alph*)]
	\addtocounter{enumi}{2}
	\item 
	My profile
	\begin{lstlisting}
# Unit limit per quarter. You can ignore this for the first
# few questions in problem 2.
minUnits 3
maxUnits 6

# These are the quarters that I need to fill. It is assumed that
# the quarters are sorted in chronological order.
register Aut2019
register Win2019

# Courses I've already taken
taken CS103
taken CS106B
taken CS107
taken CS109
taken CS140
taken CS145
taken CS161

# Courses that I'm requesting
request CS224N
request CS221
request CS228
request CS229
request CS246
request CS223A
	\end{lstlisting}
	The best schedule the course scheduler found is
	\begin{lstlisting}
Quarter  Units  Course
Aut2019  4      CS229
Win2019  4      CS246	
	\end{lstlisting}
	I think this a a reasonable academic schedule ahead.
	
\end{enumerate}
\end{document}